
\newcommand{\pluginName}{QRNG Webservice}
\newcommand{\pluginVersion}{1.0}

\input{../../DocumentationTemplate/Template}

\begin{document}

\PluginTitle{\pluginName}{\pluginVersion}

\section{Introduction}

The QRNG Webservice plug-in allows Fairmat to use a \emph{free} service hosted by the physic department of the university of Berlin as source of random numbers for the Monte-Carlo simulation Fairmat needs. 
Random numbers are generated by exploiting quantum physic properties: from the authors: \emph{We provide a new quantum random number generator (QRNG) based on the quantum randomness of photon arrival times. It promises provable and long term statistical quality, speed as well as affordability (see \url{http://qrng.physik.hu-berlin.de} for more details).}


\section{How to use the plug-in}

In order to use this plug-ins you have to follow the steps below:
\begin{itemize}

\item Register to to the QRNG page \url{http://qrng.physik.hu-berlin.de/register/} and retrieve your credentials to access to the service.
\item From the Fairmat main menu open \textbf{Settings / Fairmat Preferences / Plug-ins Preferences}, select \emph{qrn.phusik.hu-berlin.de Settings} and enter your QRNG credentials.
\item Again \textbf{Settings / Fairmat Preferences / Plug-ins Preferences} select \emph{Random Source Settings} and select \emph{qrn.phusik.hu-berlin webservice}.
\item Finally, choose the  the Random Source Support plug-ins as random generator in Fairmat: go to \textbf{Fairmat Preferences / Advanced} and select "Qrng.physik.hu-berlin.de webservice".
\end{itemize}

\bibliographystyle{unsrt}
\bibliography{../../DocumentationTemplate/bibliography}

\end{document}
